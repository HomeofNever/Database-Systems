\documentclass[11pt]{article}
\usepackage{enumerate}
\usepackage{upquote,textcomp}
\usepackage{graphicx}

\setlength{\parindent}{0cm}

\setlength{\parskip}{0.3cm plus4mm minus3mm}

\oddsidemargin = 0.2in
\textwidth = 6.5 in
\textheight = 9.8 in
\headsep = -1in

\title{Database Systems, CSCI 4380-01 \\
Homework \# 1 Answers \\
Due Thursday February 7, 2013 at 2 pm}
\date{}
\begin{document}
\maketitle

\vspace*{-0.7in}

\noindent{\bf Homework Statement.} This homework aim to teach you how
to construct complex queries using relational algebra. Please do the
parts in sequence, as they build on each other. Suppose you are given
the following database that tracks the activities of the Mars
Curiosity Rover : \\ \\
\hspace*{0.6in} {\tt RoverLocations( \underline{locationId}, name, arrivalDT, south, east, departureDT )}  \\
\hspace*{0.6in} {\tt Distances( \underline{locationid1, locationid2}, distance)} \\
\hspace*{0.6in} {\tt TestTypes ( \underline{testTypeID}, name, instrumentName, description )} \\
\hspace*{0.6in} {\tt Tests ( \underline{testDT, testTypeID}, parameters )} \\
\hspace*{0.6in} {\tt Chemistry ( \underline{chemId},  name, description )}  \\
\hspace*{0.6in} {\tt TestResults ( \underline{testDT, testTypeID, chemId}, amount, unit )} \\

Note: All datetime fields are indicated as {\tt DT} and combine both
the date and the time, e.g. a valid value would look like
\verb+ datetime '01-31-2013 08:00:00'+. You can assume that you can
check if a date time {\tt X} comes after another date time {\tt Y} by
checking whether {\tt X $>$ Y}.

The relation {\tt RoverLocations} tracks the location of the rover (in
terms of south and east degrees). Each time the rover stops, its
location has a new identifier. When the rover departs from a location,
the departure date and time is filled in. The attribute \verb+name+ is
the name given to a specific location if such a name exists. For the
rover's current location, you can assume the departureDT has a special
value {\tt Null} which is compatible with the datetime data type.

The relation {\tt Distances} tracks the actual distance between two
locations. Technically we can use this to compute speed for rover, but
we will not do this at this time. Distance information is stored only
for a couple of locations that the rover visited in sequence, meaning
that the rover visited \verb+locationid2+ right after
\verb+locationid1+.

The rover carries multiple instruments that can run many different
types of tests. Each test type is given an identifier and the detailed
information for these types are stored in the {\tt TestTypes} relation.

The relation {\tt Tests} records the many different types of tests that
the rover runs. Each test is done when the rover is stopped at a
specific location. The rover cannot run tests when moving. The tests
may have many parameters. We assume they are stored as a text field
for now.

Generally, the tests search for different chemical species (for
example specific carbon isotopes, or elements, minerals, etc.) Each
species is given a different identifier and stored in the relation
{\tt Chemistry}.

Finally, after each test, if certain chemical species are identified,
we this store in relation {\tt TestResults}. The test results include
which species are identified in a specific test and the amount
measured together with the unit.

You can see a visual explanation of the data model in the last page.

\newpage

Write the following queries using relational algebra:

{\bf Part 1.} The following queries only need a single SELECT
($\sigma$), followed by a PROJECT ($\pi$):

\begin{enumerate}  [(a)]
\item Find and return the dateTime of all tests that found at least
  0.5 parts per million (i.e. unit of \verb+ppm+) of the chemistry with id
  100.

{\bf Answer.}

\begin{eqnarray*}
Answer & = & \pi_{testDT} (\sigma_{chemid=100 \mbox{ and } unit = ppm} TestResults)
\end{eqnarray*}

\item Find the location id, south and east coordinates of all locations
  that the rover visited before \verb+datetime '11-01-2012 08:00:00'+.


{\bf Answer.}

\begin{eqnarray*}
Answer & = & \pi_{locationID, south, east} (\sigma_{arrivalDT < datetime '11-01-2012 08:00:00'} RoverLocations)
\end{eqnarray*}

\item Find all test type identifiers conducted on the instrument
  \verb+RAD+ (short for Radiation Assessment Detector).

{\bf Answer.}

\begin{eqnarray*}
Answer & = & \pi_{testTypeID} (\sigma_{instrumentName = RAD} TestTypes)
\end{eqnarray*}

\end{enumerate}

{\bf Part 2.} The following queries combine SELECT ($\sigma$), SET
operations ($\cap,\cup,\setminus$), PROJECTION ($\pi$) and RENAMING
($\rho$) as necessary:


\begin{enumerate} [(a)]
\item Find all test identifiers of test that are conducted on the
  instrument \verb+DAN+ (short for Dynamic Albedo of Neutrons) after
  \verb+datetime '1-01-2013 08:00:00'+ and have found the chemistry
  with identifier \verb+123+.

{\bf Answer.}

\begin{eqnarray*}
T1 & = & \pi_{testTypeID} (\sigma_{chemID = 123 \mbox{ and } testDB > datetime '1-01-2013 08:00:00'} TestResults) \\
T2 & = & \pi_{testTypeID} (\sigma_{instrumentName = DAN} Tests) \\
Answer & = &  T1 \cap T2
\end{eqnarray*}

\item Find all chemistry that the have not been found in any tests for
  more than 0.2 parts per million (e.g. unit {\tt ppm}). (Note: we are
  disregarding tests using different units and making no conversions.)

{\bf Answer.}

\begin{eqnarray*}
T1 & = & \pi_{chemId} (\sigma_{amount>0.2 \mbox{ and } unit = ppm} TestsResults) \\
T2 & = & \pi_{chemId} Chemistry \\
Answer & = &  T2 - T1
\end{eqnarray*}

\item Find all tests that have either found one of the chemical
  species with identifier 101, 102 or 155 with at least 0.2 parts per
  million (e.g. unit {\tt ppm}), or found both chemical species 104
  and 105 both with at least 0.4 parts per million (e.g. unit {\tt
    ppm}).

{\bf Answer.}

\begin{eqnarray*}
T1a & = & \sigma_{(chemId=101 \mbox{ or } chemId=102 \mbox{ or } chemId=155) \mbox{ and } amount=0.2 \mbox{ and } unit = ppm } TestResults \\
T1b & = & \pi_{testDT, testTypeID} (T1a) \\
T2 & = & \pi_{testDT, testTypeID} (\sigma_{chemId=104 \mbox{ and } amount=0.4 \mbox{ and } unit = ppm } TestResults) \\
T3 & = & \pi_{testDT, testTypeID} (\sigma_{chemId=105 \mbox{ and } amount=0.4 \mbox{ and } unit = ppm } TestResults) \\
T4 & = & T2 \cap T3 \\
Answer & = &  T1 \cup T4
\end{eqnarray*}

\end{enumerate}

{\bf Part 3.} The following queries combine SELECT ($\sigma$)
statements with JOIN ($\bowtie$) only, followed by a PROJECT ($\pi$):

\begin{enumerate} [(a)]
\item Find the name of all chemical species that are found by a test
  of the instrument \verb+RAD+.

{\bf Answer.}

\begin{eqnarray*}
Answer & = & \pi_{name} (Chemistry \bowtie (TestResults \bowtie (\pi_{testTypeId}(\sigma_{instrumentName=RAD} TestTypes))))
\end{eqnarray*}

Note: I am projecting out attributes in {\tt TestTypes} to remove {\tt
  TestsTypes.name} attribute. Otherwise, the resulting relation will
have two attributes called {\tt name} and the last projection would be
ambiguous and incorrect.

\item Find the identifier of all tests conducted at the location with
  coordinates \verb+south=4.5+, and \verb+east=137.4+.

{\bf Answer.}

\begin{eqnarray*}
T1 & = & \sigma_{south=4.5 \mbox{ and } east=137.4} RoverLocations \\
T2 & = & TestResults \bowtie_{testDT >= arrivalDT \mbox{ and } (testDT <= departureDT \mbox{ or } departureDT = Null)} T1 \\
Answer & = & \pi_{testDT, testTypeID} T2
\end{eqnarray*}

\end{enumerate}

{\bf Part 4.} Freeform, you decide which combination is needed.

\begin{enumerate} [(a)]
\item {\bf Bonus/Optional.} Find two locations \verb+L1,L2+ that the
  rover visited in sequence and that in both locations, it has found
  the chemical species with name \verb+methane+. Return the id of the
  locations.

{\bf Answer.}

\begin{eqnarray*}
T1 & = & \pi_{chemId} (\sigma_{name=methane} Chemistry) \\
T2 & = & \pi_{testDT} (TestResults \bowtie T1) \\
T3a & = & T2 \bowtie_{testDT >= arrivalDT \mbox{ and } (testDT < departureDT or departureDT=Null)} RoverLocations \\
T3b & = & \pi_{testDT, locationId} (T3a) \\
T4 & = & \pi_{locationId1,locationId2} (T3b \bowtie_{locationID=locationID1} Distances) \\
Answer & = & \pi_{locationId1,locationId2} (T4 \bowtie_{locationID2=locationID} T3) \\
\end{eqnarray*}

\item Find the name of all the test types that have not been performed
  before date time\\
 \verb+datetime '1-1-2013 00:00:00'+.

{\bf Answer.}

\begin{eqnarray*}
T1 & = & \pi_{testTypeID} (\sigma_{testDT < datetime '1-1-2013 00:00:00'} TestResults) \\
T2 & = & (\pi_{testTypeID} TestTypes) - T1 \\
Answer & = & \pi_{name} ( T2 \bowtie TestTypes)
\end{eqnarray*}

\end{enumerate}

\end{document}

