\documentclass[11pt]{article}
\usepackage{url}
\usepackage{enumerate}
\usepackage{wrapfig}
\usepackage{graphicx}
\usepackage{upquote,textcomp}

\usepackage{listings}
\usepackage{color}

\definecolor{mygreen}{rgb}{0,0.6,0}
\definecolor{mygray}{rgb}{0.5,0.5,0.5}
\definecolor{mymauve}{rgb}{0.58,0,0.82}

\lstset{frame=tb,
  language=sql,
  aboveskip=3mm,
  belowskip=3mm,
  showstringspaces=false,
  columns=flexible,
  basicstyle={\small\ttfamily},
  numbers=none,
  numberstyle=\tiny\color{mygray},
  keywordstyle=\color{blue},
  commentstyle=\color{mygreen},
  stringstyle=\color{mymauve},
  breaklines=true,
  breakatwhitespace=true,
  tabsize=3
}

\setlength{\parindent}{0cm}

\setlength{\parskip}{0.2cm}

\oddsidemargin = 0.2in
\textwidth = 6.5 in
\textheight = 9.8 in
\headsep = -1in

\title{Database Systems, CSCI 4380-01 \\
Homework \# 8 \\
Due Friday December 11, 2020 at 11:59:59 PM}
\date{}
\begin{document}
\maketitle

\vspace*{-0.7in}

\noindent{\bf Homework Statement.} This homework is worth 7\% of your
total grade. If you choose to skip it, Final Exam will be worth 7\%
more.


{\bf Question 1.} In this question, you are asked the reverse of
previous homework. Improve the expected run time of {\bf two queries}
from either Homework \#4 or Homework \#5 with respect to the
\verb+mediumstreaming+ database.

To facilitate this, I have created a new database for each student
with name \verb+db2_<username>+ and created a copy of mediumstreaming
database in this personal space. Please do not drop any tables as
in Homework \#6.

You can improve the run time in one of two ways:

\begin{itemize}
\item Rewrite the query solution such that it is different than your
  previous solution and my own solution, and has lower query cost. To
  do this, in your solution, list your initial query (from your
  submission), my solution and your new query. For each query, list
  the first few lines of the query plans.

\item An easier method is to create one or two indices and use my
  solution or your own solution. To document the improvement, list (a)
  the query, (b) index creation commands, (c) a few lines of the query
  plan before you create the index and (d) full query plan after you
  create the index. It is important that the full query plan in (d)
  shows that your index is being used.
  \end{itemize}

In either case, the query plans after your changes should be lower
than before for the second cost value (the time to get all the
answers). However, it is not required that the reduction in cost is
large. Even smaller improvements will be accepted.

When creating indices, consider a few simple rules of thumb that you
may have learn from the previous homework:

\begin{itemize}
  \item Relations with lots of tuples and span many disk pages are
    likely to benefit more from indices.
  \item Conditions that are more selective are likely to be more
    useful. In this include any join conditions as well.
    \item In case very selective conditions don't exist, you can still
      target index only scans. In this case, ordering of attributes is
      important.
  \end{itemize}

When I tested, I found small improvements with indices in almost all
queries, but a large improvement in only one query. Homework \#4
queries lend themselves to better improvements.

Please document this for two queries!

{\bf Question 2.} You are given the following schedules. For each
schedule, (a) check if it is serializable by drawing the conflict
graph, and (b) discuss if it is possible to obtain this schedule using
Two Phase Locking.

\verb+S1: r1(x) r2(z) r1(y) w2(w) w2(z) r3(z) w3(x) r1(w) w1(y) w3(z)+

\verb+S2: r1(x) r1(y) r2(z) r3(z) r3(x) w3(x) w2(w) w2(z) r1(w) w1(y)+

{\bf Question 3.} Suppose you are using REDO/UNDO recovery, and
the following are the contents of the log and the disk after crash.

\begin{enumerate} [(a)]
\item Which log entries should be redone, which should be undone and in
  which order?

\item Based on this information, can you conclude if FORCE or NO FORCE
  is used? Discuss.

\item Based on this information, can you conclude if STEAL or NO STEAL
  is used? Discuss.
\end{enumerate}

\begin{tabular}{ll}
LOG: 
\begin{tabular}{l|l} 
LSN & Entry \\ \hline
100 & T1 update P2 \\
101 & T2 update P1 \\
102 & T2 commit \\ 
103 & T3 update P1 \\
104 & T3 update P3 \\
105 & T1 update P4 \\
106 & T3 commit \\ 
\end{tabular}
&
Data pages: 
\begin{tabular}{l|l} 
Data Page & LSN of Last recorded log entry \\ \hline
P1 & 101 \\ 
P2 & 100 \\
P3 & 104 \\
P4 & - \\
\end{tabular}
\end{tabular}

\newpage

{\bf SUBMISSION INSTRUCTIONS.}

Submit a single text or PDF file that documents your answers to the
homework.

\end{document}
